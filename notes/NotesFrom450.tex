\title{Notes from CS 450}
\author{
        Daniel Craig \\
                Department of Computer Science\\
        BYU Idaho - Class of 2019\\
}
\date{\today}

\documentclass[12pt]{article}
\usepackage{tikz}
\usetikzlibrary{arrows, calc}


\begin{document}
\maketitle

\section{Notes from Chapter 1}
\begin{verse}
I have not read Chapter 1 yet.
\end{verse}

\section{Notes from Chapter 2}
\begin{verse}
``Accuracy is defined as the sum of the number of true positives and true negatives divided by the total number of examples."
\end{verse}

\begin{equation}
Accuracy = \frac{TP + TN}{TP + FP + TN + FN}
\end{equation}

\begin{verse}
``Sensitivity (also known as the true positive rate) is the ratio of the number of correct positive examples to the number classified as positive"
\end{verse}

\begin{equation}
Sensitivity = \frac{TP}{TP + FN}
\end{equation}

\begin{verse}
``specificity is the same ratio for negative examples"
\end{verse}

\begin{equation}
Specificity = \frac{TN}{TN + FP}
\end{equation}

\begin{verse}
``Precision is the ratio of correct positive examples to the number of actual positive examples"
\end{verse}

\begin{equation}
Precision = \frac{TP}{TP + FP}
\end{equation}

\begin{verse}
Recall is the ratio of the number of correct positive examples out of those that were classified as positive, which is the same as sensitivity.
\end{verse}

\begin{equation}
\frac{TP}{TP + FN}
\end{equation}

\begin{verse}
Bayes rule is the most important equation in machine learning, and it is as follows:
\end{verse}

\begin{equation}
P(C_i|X_j) = \frac{P(X_j|C_i)P(C_i)}{P(X_j1)}
\end{equation}

\begin{verse}
``Bayes rule relates the posterior probability $$P(C_i|X_j)$$ with the prior 
probability $$P(C_i)$$ and class conditional probability $$P(X_j|C_i)$$. The denominator (the term on the bottom of the fraction) acts to normalise everything, so that all the probabilities sum to 1."
\end{verse}

\section{Notes from Chapter 12}
\begin{verse}
The highest information gain is achieved by selecting for the lowest entropy. See figure 1.
\end{verse}

\begin{figure}
\centering
\begin{tikzpicture}
	\draw[->] (0,0) -- (5,0) node[right] {$x$};
	\draw[->] (0,0) -- (0,5) node[above] {$y$};
	\draw[scale=5,domain=.01:1,smooth,variable=\x,gray] plot ({\x},{-\x*log2(\x)-((1-\x)*log2(1-\x))});
	
	\path
	coordinate (top) at (.5,1)
	coordinate (bottom) at (.5,0);
\end{tikzpicture}
\caption{The Binary Entropy Function} \label{fig:M1}
\end{figure}

\begin{verse}
Entropy is defined as the following:
\end{verse}

\begin{equation}
Entropy(p) = -\sum_i{p_i\log_2{p_i}}
\end{equation}

\begin{verse}
Information gain is defined as "the entropy of the whole set minus the entropy when a particular feature is chosen."
\end{verse}

\begin{equation}
Gain(S,F)=Entropy(S) -\sum_{f\in values(F)}{\frac{\mid S_f \mid}{\mid S \mid}Entropy(S_f)}
\end{equation}

\end{document}
  